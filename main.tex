\documentclass{report}

\usepackage[utf8]{inputenc}
\usepackage[brazil]{babel}
\usepackage[colorlinks]{hyperref}
\usepackage{amssymb}
\usepackage{graphicx}
\usepackage{color}
\usepackage{titlesec}
\usepackage{fancyhdr}
\usepackage{enumitem}
\newlist{todolist}{itemize}{2}
\setlist[todolist]{label=$\square$}
\usepackage{pifont}
\newcommand{\cmark}{\ding{51}}
\newcommand{\xmark}{\ding{55}}
\newcommand{\done}{\rlap{$\square$}{\raisebox{2pt}{\large\hspace{1pt}\cmark}}%
\hspace{-2.5pt}}
\newcommand{\wontfix}{\rlap{$\square$}{\large\hspace{1pt}\xmark}}

\begin{document}

\title{Estudo Comparativo de Métodos de Análise de Riscos de Segurança da Informação}
\author{Deise Freire}
\date{\today}

\maketitle

\fancypagestyle{plain}{
\fancyhead[L]{\color{black}\sffamily\bfseries Perguntas e Respostas}
}
\pagestyle{plain}

\begin{itemize}
  \item Qual a importância da análise de risco de TI para pequenas empresas?
  \begin{todolist}
  \item[\done] A análise de risco de TI é crucial para identificar e mitigar os riscos que podem prejudicar os sistemas de informação e os dados de uma empresa.
  \end{todolist}
\end{itemize}

\begin{itemize}
  \item Qual o problema com a seleção de métodos de análise de risco em pequenas empresas?
  \begin{todolist}
  \item[\done] Muitas vezes, as pequenas empresas escolhem métodos inadequados, o que resulta em análises de risco deficientes.
  \end{todolist}
\end{itemize}

\begin{itemize}
  \item Como uma estratégia adequada de análise de risco pode ajudar?
  \begin{todolist}
  \item[\done] Uma estratégia adequada pode ajudar as pequenas empresas a escolher o método de análise de risco mais adequado às suas necessidades.
  \end{todolist}
\end{itemize}

\begin{itemize}
  \item Quantos métodos de análise de risco existem?
  \begin{todolist}
  \item[\done] Existem muitos métodos de análise de risco disponíveis, cada um com suas características e finalidades específicas.
  \end{todolist}
\end{itemize}

\begin{itemize}
    \item Qual a melhor maneira de escolher entre os métodos de análise de risco?
    \begin{todolist}
        \item[\done] A melhor maneira é compará-los com base em critérios como abordagem, nível de conhecimento necessário, parâmetros de entrada e resultado.
    \end{todolist}
\end{itemize}

\begin{itemize}
    \item Qual o benefício de usar critérios objetivos na comparação de métodos?
    \begin{todolist}
        \item[\done] Permite que as empresas comparem diferentes metodologias de forma imparcial e tomem uma decisão mais informada sobre qual a melhor. 
    \end{todolist}
\end{itemize}


\begin{itemize}
    \item Qual a importância da análise de risco na gestão de negócios?
    \begin{todolist}
        \item[\done] A análise de risco é crucial para identificar e avaliar os riscos que podem afetar o sucesso de uma organização. 
    \end{todolist}
\end{itemize}


\begin{itemize}
    \item Quantas metodologias de análise de risco existem?
    \begin{todolist}
        \item[\done] Existem diversas metodologias disponíveis, cada uma com suas características e vantagens. 
    \end{todolist}
\end{itemize}


\end{document}
